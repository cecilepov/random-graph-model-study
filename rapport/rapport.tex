% *-----------------------------------------------------------*
% *         Page de Garde (basé sur le modèle UPMC)
% * Dernière modif le : 2012/02/13
% *-----------------------------------------------------------*
% * README: 
% *   - se compile avec pdflatex
% *   - l'entête de ce fichier sert pour générer une page
% *     unique. Ce texte latex peut s'inclure (partie comprise 
% *     entre les lignes et  ) dans un document existant.
% *-----------------------------------------------------------*

% - - - - - - - entête pour avoir une page unique 
\documentclass[11pt]{report}
\usepackage[english]{babel}
%\usepackage[latin1]{inputenc}
\usepackage{graphicx}

%\usepackage{graphicx}
\usepackage{subfigure}
\usepackage{tabularx,tabulary}
\usepackage{svg}
\usepackage{blindtext}
\usepackage{hyperref} % clickable table of contents
\usepackage{fancyhdr}
\usepackage{amsmath} % equations

\hypersetup{ % hyperref - clickable table of contents
    colorlinks,
    citecolor=black,
    filecolor=black,
    linkcolor=black,
    urlcolor=black
}

\setlength{\textwidth}{16cm}
\setlength{\textheight}{25cm}
\setlength{\oddsidemargin}{0cm}
\setlength{\topmargin}{-2cm}
\title{Mon super titre}     
\date{ma superbe date} 
\author{Cécile Pov}





 % NO BREAKKING WORDS !
\tolerance=1
\emergencystretch=\maxdimen
\hyphenpenalty=10000
\hbadness=10000

\pagestyle{fancy}


% ENTETE ET PIED DE PAGE 
%\newcommand{\markedsection}[2]{\section[#2]{#2%
%\sectionmark{#1}}
%\sectionmark{#1}}

%\newcommand{\markedsubsection}[2]{\subsection[#2]{#2%
%\subsectionmark{#1}}
%\subsectionmark{#1}}





\begin{document}

% - - - - - - - début de la page 
\thispagestyle{empty}

%\includegraphics[scale=0.12]{upmc-logo.png}
\begin{figure}[t]

\minipage{0.30\textwidth}
%  \includegraphics[width=\linewidth]{./img/logo_sorbonne.png}
  \includesvg[width=\linewidth]{./img/logo_esiee}
\endminipage\hfill
\minipage{0.25\textwidth}
   %\hspace*{-2.2cm}
  \includegraphics[width=1.5\textwidth]{./img/logo_sorbonne.png}
\endminipage\hfill
\minipage{0.3\textwidth}
\hspace*{2.2cm}
\vspace*{0.3cm}
  \includesvg[width=0.6\linewidth]{./img/logo_lip6}
\endminipage
\end{figure}

{\large

\vspace*{1cm}

\begin{center}

{\bf MASTER DEGREE INTERNSHIP RESEARCH PAPER}

\vspace*{0.4cm}

Computer Science Department 
%\\ [2ex]
%{\bf Informatique}\ \\ 

\vspace*{0.1cm}

ESIEE Paris

\vspace*{1cm}

\rule{\linewidth}{0.5pt}
{\Large {\sc A study of a random model preserving maximal bicliques in bipartite graphs}}
\rule{\linewidth}{1pt}

\vspace*{0.5cm}
By\ \\


\vspace*{0.5cm}


{\Large {\bf Cécile POV}}

\vspace*{1cm}


At

\vspace*{0.5cm}

{\Large {\bf Complex Networks team, LIP6} \\
Sorbonne University and French National Center for Scientific Research (UMR 7606 Sorbonne University - CNRS)}



%Pour obtenir le grade de \ \\[1ex]
%{\bf DOCTEUR de l'UNIVERSIT\'E PIERRE ET MARIE CURIE} \ \\

\vspace*{1cm}

\end{center}

%\flushleft{Sujet de la th\`ese :\ \\
%\ \\
%{\Large {\bf Le titre de ma Thèse \\ }}
%  

\vspace*{1.5cm} 
\flushleft{Submitted in October 2020}\\[2ex]
\flushleft{Defense Commitee composed of:}\\[1ex]
\flushleft{\begin{tabularx}{\textwidth}{lXX}
Mr Fabien TARISSAN & Complex Networks - LIP6, CNRS & Internship supervisor\\
Mr Lionel TABOURIER  &  Complex Networks - LIP6 & Internship supervisor \\
Mr Nabil Hassan MUSTAFA  & ESIEE Paris, LIGM & University supervisor
\end{tabularx}}


}
% - - - - - - - fin de la page 
 

\tableofcontents


\newpage
\section*{Abstract}
Many large real-world networks have a 

Keywords: bipartite graphs, configuration model, tripartite encoding, maximal bicliques


%\markedsection{Introduction}{Introduction}
\markedsubsection{Erdős–Rényi model}{Erdős–Rényi model}


Number of possible edges (without self-loops):
$$\binom{n}{k} = \dfrac{n!}{2!(n-2)!} = 
\dfrac{n(n-1)(n-2)!}{2(n-2)!} = \dfrac{n(n-1)}{2}$$

The different instances of $G(n,p)$ appears with different frequencies, so it is a distribution of graphs, not a family.

$$P(G)=p^{m}\times(1-p)^{M-m}$$

$$P(m)=\binom{n}{k}\times  p^{m}\times(1-p)^{M-m}$$

\markedsubsection{}{Binomial distribution}

$$P(2) = \binom{\frac{3 \times 2}{2}}{2} \times \binom{1}{2}^{2}$$

\markedsubsection {} {Clustering coeff}

The global clustering coefficient $CC(G)$ measures the cliquishness of a graph $G$, i.e.how well the vertices are connected together. It gives a score between 0 and 1: the closer to one the score is, the better the vertices are connected together.

The global clustering coefficient is computed as follows (in general we consider only vertices  with degree > 1):
$$CC(G) = mean(CC(v)) \ \  v \in G $$ 


The local clustering coefficient for a vertex $v$ is equal to:
$$CC(v) = \dfrac{2Nv}{kv( kv-1) }$$

With : 

\begin{itemize}
    \item $Kv$ the degree of the vertex $v$;
    \item $Nv$ the number of edges between neighbors of v;
\end{itemize}



CC(V) is a "ratio":

\begin{itemize}
    \item $2Nv$ is the actual number of existing interconnections (multiplied by 2 because an edge links 2 nodes);
    \item $Kv(Kv-1)$ is the maximum number of possible interconnections
\end{itemize}


There is at most $Kv(Kv-1)/2$ edges between neighbors.


\markedsubsection{}{Assortativity}


Assortativity measures the preference for vertices with same degree to connect together. In assortative networks, hubs are connected together ("big ones with big ones and small ones with small ones); on the contrary, high degree nodes tends to connect to low degree nodes in disassortative networks.

Pearson correlation coefficient measures the linear correlation between two variables $X$ and $Y$.
In our case, the assortativity coefficient is the Pearson correlation coefficient of degree between pairs of linked nodes.

We consider:

\begin{itemize}
    \item $x$ as the initial extremity of the node;
    \item $y$ as the terminal extremity.
\end{itemize}


Since we are working with undirected graph in this part on the project, we must consider each edge twice (edges $(i,j)$ and $(j,i)$), so in our case we have 
$2 \times \textrm{total\_nb\_edges}$ samples.

$r$ has a value between $+1$ and $-1$ : 
\begin{itemize}
	\item $r=1$  : total positive linear correlation between $x$ and $y$;
	\item $r=0$  : no linear correlation between $x$ and $y$;
	\item $r=-1$ : total negative linear correlation between $x$ and $y$.
\end{itemize}

$$\newcommand \xdiff    {(x_{i}-\overline {x})}
\newcommand   \ydiff    {(y_{i}-\overline {y})}
\newcommand   \sumassort{\sum_{i=1}^{n}}
r_{xy} = \dfrac{\sumassort \xdiff \ydiff } {\sqrt{\sumassort \xdiff^{2} \sumassort \ydiff^{2}}}$$

With:
- n the number of samples;
- $\overline{x}$ the mean of all $x_{i}$;
- $\overline{y}$ the mean of all $y_{i}$.


\markedsubsection{}{Min-Max clustering}

$$cc_{\bullet}(u,v)=\dfrac{\mid N(u)\cap N(v) \mid}
						  {\mid N(u)\cup N(v) \mid} = \dfrac{common}{all}$$
						  
$$cc_{\bullet}(u,v)=\dfrac{\mid N(u)\cap N(v) \mid}
						  {min(\mid N(u)\cup N(v) \mid)}$$
		
$$cc_{\bullet}(u,v)=\dfrac{\mid N(u)\cap N(v) \mid}
						  {max(\mid N(u)\cup N(v) \mid)}$$				  

\markedsubsection{}{Configuration Model}

Probability that 2 nodes are connected:
$$p_{ij} = \dfrac{k_{i}k_{j}}{2m-1}$$

\markedsubsection{}{Damaschke}
$\sigma(A) = \cap_{v\in A} N(v)$
$$\phi(A) = \sigma(\sigma(A))$$

$$\phi(\phi(A)) = \phi(A)$$


\subsection{Redundancy}

$$rc(v)= \dfrac{\mid \{\{u,w\} \subseteq N(v), \exists v' \neq v, ()   ff \mid}{•}$$
\end{document}



%\documentclass{article}
%
%\usepackage{graphicx}
%\usepackage{subfigure}
%\usepackage{tabularx,tabulary}
%\usepackage{svg}
%\usepackage{blindtext}
%
%
%\begin{document}
%
%
%%\begin{titlepage}
% 
%
%\begin{figure}[t]
%
%
%\minipage{0.32\textwidth}
%%  \includegraphics[width=\linewidth]{./img/logo_sorbonne.png}
%  \includesvg[width=\linewidth]{./img/logo_esiee}
%\endminipage\hfill
%\minipage{0.32\textwidth}
%  \includegraphics[width=1.5\textwidth]{./img/logo_sorbonne.png}
%\endminipage\hfill
%%\minipage{0.32\textwidth}%
%%  \includesvg[width=0.6\linewidth]{./img/logo_lip6}
%%\endminipage
%\end{figure}
%
%
%\title{\rule{\linewidth}{1pt}
%A study of a random model preserving maximal bicliques in bipartite graphs
%\rule{\linewidth}{1pt}}
%
%
%
%
%
%%\end{titlepage}
%
%
%\end{document}


